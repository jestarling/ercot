%----------------------------------------------------------------------------------------
%	PACKAGES AND OTHER DOCUMENT CONFIGURATIONS
%----------------------------------------------------------------------------------------

\documentclass{article}

%SPENCER NICE FONTS
\usepackage{mathpazo}
\usepackage{avant}
\usepackage{inconsolata}

\usepackage{fancyhdr} % Required for custom headers
\usepackage{lastpage} % Required to determine the last page for the footer
\usepackage{extramarks} % Required for headers and footers
\usepackage[usenames,dvipsnames]{color} % Required for custom colors
\usepackage{graphicx} % Required to insert images
\usepackage{listings} % Required for insertion of code
\usepackage{amsmath}  % Required for \text{} function
%\usepackage{couriernew} % Required for the courier font
\usepackage{pdfpages} %Required to embed pdf file.
\usepackage{enumerate} % Required for enumerating with letters
\usepackage{amssymb} %Required for QED symbols.
\usepackage{bbm}  %Required for indicator function.
\usepackage{float}	%Required for strict figure placement [H].
\usepackage{enumitem}% Required for square brackets in enumerate lists. http://ctan.org/pkg/enumitem

%CUSTOMIZE INDENTS:
\newcommand{\myindent}{\hspace*{.25cm}}
\setlength{\parindent}{0pt}

%ARGMIN:
\DeclareMathOperator*{\argmin}{argmin}

%ARGMAX:
\DeclareMathOperator*{\argmax}{argmax}

% Margins
\topmargin=-0.45in
\evensidemargin=0in
\oddsidemargin=0in
\textwidth=6.5in
\textheight=9.5in
\headsep=0.25in

\linespread{1.1} % Line spacing

% Set up the header and footer
\pagestyle{fancy}
\lhead{J. Starling} % Top left header
\chead{} % Top center head
\rhead{Page\ \thepage\ of\ \protect\pageref{LastPage}} % Top right header
%\lfoot{\lastxmark} % Bottom left footer
\cfoot{} % Bottom center footer
%\rfoot{Page\ \thepage\ of\ \protect\pageref{LastPage}} % Bottom right footer
\renewcommand\headrulewidth{0.4pt} % Size of the header rule
%\renewcommand\footrulewidth{0.4pt} % Size of the footer rule

\setlength\parindent{0pt} % Removes all indentation from paragraphs

%----------------------------------------------------------------------------------------
%	CODE INCLUSION CONFIGURATION
%----------------------------------------------------------------------------------------

\definecolor{MyDarkGreen}{rgb}{0.0,0.4,0.0} % This is the color used for comments
\lstloadlanguages{R} % Load R syntax for listings, for a list of other languages supported see: ftp://ftp.tex.ac.uk/tex-archive/macros/latex/contrib/listings/listings.pdf
\lstset{language=R, % Use R in this example
        frame=single, % Single frame around code
        basicstyle=\small\ttfamily, % Use small true type font
        keywordstyle=[1]\color{Blue}, % Perl functions bold and blue
        keywordstyle=[2]\color{Purple}, % Perl function arguments purple
        keywordstyle=[3]\color{Blue}\underbar, % Custom functions underlined and blue
        identifierstyle=, % Nothing special about identifiers                                         
        commentstyle=\usefont{T1}{pcr}{m}{sl}\color{MyDarkGreen}\small, % Comments small dark green courier font
        stringstyle=\color{Purple}, % Strings are purple
        showstringspaces=false, % Don't put marks in string spaces
        tabsize=4, % 5 spaces per tab
        %
        % Put standard Perl functions not included in the default language here
        morekeywords={rand},
        %
        % Put Perl function parameters here
        morekeywords=[2]{on, off, interp},
        %
        % Put user defined functions here
        morekeywords=[3]{test},
       	%
        morecomment=[l][\color{Blue}]{...}, % Line continuation (...) like blue comment
        numbers=left, % Line numbers on left
        firstnumber=1, % Line numbers start with line 1
        numberstyle=\tiny\color{Blue}, % Line numbers are blue and small
        stepnumber=5 % Line numbers go in steps of 5
}

% Creates a new command to include a perl script, the first parameter is the filename of the script (without .pl), the second parameter is the caption
\newcommand{\rscript}[2]{
\begin{itemize}
\item[]\lstinputlisting[caption=#2,label=#1]{#1.r}
\end{itemize}
}

%----------------------------------------------------------------------------------------
%	DOCUMENT STRUCTURE COMMANDS
%	Skip this unless you know what you're doing
%----------------------------------------------------------------------------------------

% Header and footer for when a page split occurs within a problem environment
\newcommand{\enterProblemHeader}[1]{
\nobreak\extramarks{#1}{#1 continued on next page\ldots}\nobreak
\nobreak\extramarks{#1 (continued)}{#1 continued on next page\ldots}\nobreak
}

% Header and footer for when a page split occurs between problem environments
\newcommand{\exitProblemHeader}[1]{
\nobreak\extramarks{#1 (continued)}{#1 continued on next page\ldots}\nobreak
\nobreak\extramarks{#1}{}\nobreak
}

\setcounter{secnumdepth}{0} % Removes default section numbers
\newcounter{homeworkProblemCounter} % Creates a counter to keep track of the number of problems

\newcommand{\homeworkProblemName}{}
\newenvironment{homeworkProblem}[1][Problem \arabic{homeworkProblemCounter}]{ % Makes a new environment called homeworkProblem which takes 1 argument (custom name) but the default is "Problem #"
\stepcounter{homeworkProblemCounter} % Increase counter for number of problems
\renewcommand{\homeworkProblemName}{#1} % Assign \homeworkProblemName the name of the problem
\section{\homeworkProblemName} % Make a section in the document with the custom problem count
\enterProblemHeader{\homeworkProblemName} % Header and footer within the environment
}{
\exitProblemHeader{\homeworkProblemName} % Header and footer after the environment
}

\newcommand{\problemAnswer}[1]{ % Defines the problem answer command with the content as the only argument
\noindent\framebox[\columnwidth][c]{\begin{minipage}{0.98\columnwidth}#1\end{minipage}} % Makes the box around the problem answer and puts the content inside
}

\newcommand{\homeworkSectionName}{}
\newenvironment{homeworkSection}[1]{ % New environment for sections within homework problems, takes 1 argument - the name of the section
\renewcommand{\homeworkSectionName}{#1} % Assign \homeworkSectionName to the name of the section from the environment argument
\subsection{\homeworkSectionName} % Make a subsection with the custom name of the subsection
\enterProblemHeader{\homeworkProblemName\ [\homeworkSectionName]} % Header and footer within the environment
}{
\enterProblemHeader{\homeworkProblemName} % Header and footer after the environment
}

%----------------------------------------------------------------------------------------
%	SUBSECTION NUMBERING
%----------------------------------------------------------------------------------------

%----------------------------------------------------------------------------------------
%	NAME AND CLASS SECTION
%----------------------------------------------------------------------------------------

\newcommand{\hmwkTitle}{Forecasting Texas Energy Grid Demand \\ with \\ Dynamic Linear Models} % Assignment title
\newcommand{\hmwkDueDate}{May\ 1,\ 2017} % Due date
\newcommand{\hmwkClass}{SDS 383D Modeling 2} % Course/class
\newcommand{\hmwkClassTime}{} % Class/lecture time
\newcommand{\hmwkClassInstructor}{Professor James Scott} % Teacher/lecturer
\newcommand{\hmwkAuthorName}{Jennifer Starling } % Your name

%----------------------------------------------------------------------------------------
%	TITLE PAGE
%----------------------------------------------------------------------------------------

\title{
\vspace{2in}
\textmd{\textbf{\hmwkTitle}}\\
\normalsize\vspace{0.1in}\small{\hmwkDueDate}\\
\vspace{0.1in}\large{\textit{\hmwkClassInstructor\ }}
\vspace{3in}
}

\author{\textbf{\hmwkAuthorName}}
\date{} % Insert date here if you want it to appear below your name

%----------------------------------------------------------------------------------------

\begin{document}

\maketitle
\thispagestyle{empty} %hides page number on title page.

%Includes pdf document at beginning:
%\includepdf[pages=-,pagecommand={},width=\textwidth]{Homework_3.pdf}

%%%%%%%%%%%%%%%%%%%%%%%%%%%%%%%%%%%%%%%%%%%%%%%%%%%%%%%%%%%%%%%%%%%%
%%                     PART 1                                     %%
%%%%%%%%%%%%%%%%%%%%%%%%%%%%%%%%%%%%%%%%%%%%%%%%%%%%%%%%%%%%%%%%%%%%
\newpage
\section{1. Introduction}






%%%%%%%%%%%%%%%%%%%%%%%%%%%%%%%%%%%%%%%%%%%%%%%%%%%%%%%%%%%%%%%%%%%%
%%                     PART 2                                     %%
%%%%%%%%%%%%%%%%%%%%%%%%%%%%%%%%%%%%%%%%%%%%%%%%%%%%%%%%%%%%%%%%%%%%
\newpage
\section{2. Data and Methods}

\subsection{2.1 Energy grid load data}

Our data set consists of hourly temperature, business hour and energy grid load data for Texas, measured from Dec. 31, 2009 to Aug. 15, 2016.  Energy grid load is measured in separately for eight zones in Texas, which are then aggregated to calculate total load for the entire Texas grid.  We fit separate models for each zone to accomodate each zone's unique matrix of covariates.\\

Temperature is measured following the current methodology used by ERCOT to forecast Texas energy demand.  This model calculates each zone temperature at time $t$ as the weighted average of selected weather stations. Missing weather station observations are linearly interpolated.

% latex table generated in R 3.3.2 by xtable 1.8-2 package
% Mon Apr 24 08:19:24 2017
\begin{table}[ht]
	\caption{Temperature Weighting by Zone}
\centering
\begin{tabular}{rllr}
  \hline
  Zone & Weather Station & Weight \\ 
  \hline
  North & KSPS & 0.50 \\ 
   North & KPRX & 0.50 \\ 
   North Central & KDFW & 0.50 \\ 
   North Central & KACT & 0.25 \\ 
   North Central & KMWL & 0.25 \\ 
   East & KTYR & 0.50 \\ 
   East & KLFK & 0.50 \\ 
   Far West & KINK & 0.50 \\ 
   Far West & KMAF & 0.50 \\ 
   West & KABI & 0.40 \\ 
   West & KSJT & 0.40 \\ 
   West & KJCT & 0.20 \\ 
   South Central & KAUS & 0.50 \\ 
   South Central & KSAT & 0.50 \\ 
   Coast & KLVJ & 0.50 \\ 
   Coast & KGLS & 0.30 \\ 
   Coast & KVCT & 0.20 \\ 
   Southern & KCRP & 0.40 \\ 
   Southern & KBRO & 0.40 \\ 
   Southern & KLRD & 0.20 \\ 
   \hline
\end{tabular}
\end{table}

\subsection{2.2 Dynamic Linear Models}
We fit a univariate dynamic linear model as described by West \& Harrison [1]. These models are frequently used for forecasting time-series data. Under the Gaussian setting, these models are computationally tractable via the well-known Kalman Filter.  Taking advantage of the Gibbs Sampler allows us to leverage the Gaussian updates while accounting for uncertainty regarding the unknown variance and covariance terms.  \emph{(Add info about other methods which have been used to do this - a bit of general background?  Perhaps not in this section, but in intro?)}

\subsection{2.3 Model-Fitting Details}
We fit a univariate dynamic linear model as described by West \& Harrison [1], with $y_t$ describing energy grid load at discrete hourly intervals $t$.  The system and state equations can be written as
\begin{align*}
	y_t &= F_t \theta_t + \epsilon_t \text{, with } \epsilon_t \sim N_1(0,V_t) \\
	%
	\theta_t &= G_t \theta_{t-1} + \eta_t \text{, with } \eta_t \sim N_p(0,W_t)
\end{align*}

We set $G_t$ to the identity matrix, and assume error variances $V_t$ and $W_t$ are time-invariant.  The simplified model can be written as follow.  Note that V is scalar and W is a $p \times p$ matrix.
\begin{align*}
	y_t &= F_t \theta_t + \epsilon_t \text{, with } \epsilon_t \sim N_1(0,V) \\
	%
	\theta_t &= \theta_{t-1} + \eta_t \text{, with } \eta_t \sim N_p(0,W)
\end{align*}

The matrix $F_t$ describes intercept and time-varying covariates, where $\gamma_t$ represents temperature at time $t$, $\beta_t$ is an indicator for business hour at time $t$.  Specifically, $\beta_t=1$ indicates a business hour, while $\beta_t=0$ indicates nights, weekends and holidays. An autoregressive term $y_{t-1}$ is the previous hour's grid load, and $h_{1t},\hdots,h_{24t}$ are indicators representing the hours of the day.
\begin{align*}
	F_t = \left[\begin{smallmatrix} 1 \hspace{.25cm}
	\gamma_t 	\hspace{.25cm}	
	\gamma_t^2	\hspace{.25cm}
	\beta_t  	\hspace{.25cm}
	y_{t-1} 	\hspace{.25cm}
	h_{1t}, 	\hspace{.25cm}
	\hdots, 	\hspace{.25cm}
	h_{24t}\end{smallmatrix}\right]
\end{align*}

There are a variety of methods available for estimating the unknown variances $V$ and $W$.  Discount factor methods are popular, as is the \emph{d-inverse-gamma} prior.  We adopt the latter method, which is described as the most popular method by Petris [2].  The \emph{d-inverse-gamma} prior is easily written in terms of precisions, such that
\begin{align*}
	V &= \phi_y ^{-1} \\
	W &= diag\left(\phi_{\theta,1} ^{-1} , \hdots , \phi_{\theta,p} ^{-1} \right)
\end{align*}
We call the vector of precisions $\psi = \left(\phi_y ^{-1}, \phi_{\theta,1} ^{-1} , \hdots , \phi_{\theta,p} ^{-1} \right)$ and assign the $d=(p+1)$ terms independent Gamma priors.  Then the prior on the vector of variances is the product of $d$ inverse Gamma densities, hence the name of the method.  We parameterize the hyperparameters in terms of guesses of the means and variances of the unknown precisions, as described by Petris[2].  Let $E(\phi_y) = a_y$, and $E(\phi_{\theta,i}) = a_{\theta_i}$.  Let prior uncertainty by expressed as $Var(\phi_y) = b_y$ and $Var(\phi_{\theta,i}) = b_{\theta,i}$ for $i = 1,\hdots,p$.  We can then parameterize our priors as
\begin{align*}
	\phi_y &\sim Ga(\alpha_y, \beta_y) \text{, with } \alpha_y = \frac{a_y^2}{b_y} \text{, } \beta_y = \frac{a_y}{b_y}\\
	%
	\phi_{\theta,i} &\sim Ga(\alpha_{\theta,i},\beta_{\theta,i}), 
		\text{ with } \alpha_{\theta,i} = \frac{a_{\theta,i}^2}{b_{\theta,i}} \text{, }
		\beta_{\theta,i} = \frac{a_{\theta,i}}{b_{\theta,i}} \text{, } i = 1,\hdots,p
\end{align*}

Given observations $y_{t:T}$ we can write the joint posterior of $\theta_{0:T}$ 
and $\psi = \left(\phi_y ^{-1}, \phi_{\theta,1} ^{-1} , \hdots , \phi_{\theta,p} ^{-1} \right)$ as
\begin{align*}
	p\left(\theta_{0:T}, \psi | y_{1:T} \right) &= 
	p\left(y_{1:T} | \theta_{0:T}, \psi \right) \cdot 
	p\left(\theta_{0:T} | \psi  \right) \cdot 
	p\left(\psi \right) \\
	%
	&=\prod_{t=1}^{T} p\left(y_t | \theta_t, \phi_y \right)
	\cdot 
	\prod_{t=1}^{T} p\left(\theta_t | \theta_{t-1}, \phi_{\theta,1}, \hdots, \phi_{\theta,p} \right)
	\cdot
	p\left(\theta_0 \right)
	\cdot
	p\left(\phi_y \right)
	\cdot \prod_{t=1}^{T} p(\phi_{\theta,i})
\end{align*}

We then derive the full conditional of $\phi_y$.
\begin{align*}
	p\left(\phi_y | \hdots \right) &\propto
	\prod_{t=1}^{T} p\left(y_t | \theta_t, \phi_y \right) \cdot p\left(\phi_y \right) \\
	%
	&\propto \theta_y^{\frac{T}{2}} 
		\exp\left[-\frac{\phi_y}{2}\sum_{t=1}^{T} \left(y_t - F_t\theta_t \right)^2 \right]
	\cdot
	\phi_y^{\alpha_y-1}
	\exp\left[-\phi_y \beta_y \right] \\
	%
	&= \phi_y^{\alpha_y + \frac{T}{2}-1}
	\exp\left[-\frac{\phi_y}{2}\sum_{t=1}^{T} \left(y_t - F_t\theta_t \right)^2 + \beta_y \right]
\end{align*}
which we recognize as the gamma kernel.  The full conditional of $\phi_y$ is
\begin{align*}
	\phi_y &\sim Ga\left( 
		\alpha_y + \frac{T}{2} \text{, }
		\frac{1}{2}\sum_{t=1}^{T} \left(y_t - F_t\theta_t \right)^2 + \beta_y
	\right)
\end{align*}

The full conditionals of the $\phi_{\theta,i}$ for $i = 1,\hdots,p$ are derived similarly.  Since W is a diagonal matrix, the full conditionals can be written in terms of a sum over the $p$ $\theta_t$ terms, so that each $\phi_{\theta,i}$ depends only on its corresponding $\theta_{i,t}$.
\begin{align*}
	p\left(\phi_{\theta,i} | \hdots \right) &\propto
	%
	p\left(\phi_{\theta,i} \right)
	\cdot
	\prod_{t=1}^{T} p\left(\theta_t | \theta_{t-1}, \phi_{\theta,1},\hdots,\phi_{\theta,p} \right)\\
	%
	&\propto \phi_{\theta_1}^{\frac{T}{2}} \cdot \cdot \cdot \phi_{\theta_p}^{\frac{T}{2}}
	%
	\exp\left[-\frac{1}{2} \sum_{i=1}^{p} \sum_{t=1}^{T} \phi_{\theta,i} \left(\theta_{i,t} - 
		\left(G_t \theta_{t-1} \right)_i \right)^2
	\right]\\
	%
	&= 	\exp\left[-\frac{1}{2} \sum_{i=1}^{p} \sum_{t=1}^{T} \phi_{\theta,i} \left(\theta_{i,t} - 
		\theta_{i,t-1} \right)^2
	\right]
\end{align*}

Then for the single $\phi_{\theta,i}$, all other $\phi_{\theta,j}$ terms become part of the constant of proportionality, leaving
\begin{align*}
	p\left(\phi_{\theta,i} | \hdots \right) &\propto
	%
	\phi_{\theta,i}^{\left(\alpha_{\theta,i}-1 \right)}
	\exp\left[-\phi_{\theta,i}\beta_ {\theta,i}\right]
	\cdot
	\phi_{\theta,i}^{\frac{T}{2}}
	\exp\left[-\frac{\phi_{\theta,i}}{2} \sum_{t=1}^{T}
		\left(\theta_{i,t} - \theta_{i,t-1} \right)^2
	 \right]
\end{align*}
which we again recognize as the gamma kernel.  The full conditional of $\phi_{\theta,i}$ is
\begin{align*}
	p\left(\phi_{\theta,i} | \hdots \right) &\sim
	%
	Ga\left(
		\alpha_{\theta,i} + \frac{T}{2} \text{, } 
		\beta_{\theta,i} + \frac{1}{2}\sum_{t=1}^{T}
		\left(\theta_{i,t} - \theta_{i,t-1} \right)^2
	\right)
\end{align*}

We implement this model using a Gibbs Sampler, where the update for the unknown states $\theta_{0:T}$ is performed using the Forward Filtering Backward Sampling algorithm (FFBS), as detailed in Carter \& Kohn [3].  Borrowing notation from Petris [2], the FFBS algorithm for sampling the states is
\begin{align*}
	& \text{(1) Run the Kalman Filter.} \\
	%
	& \text{(2) Draw } \theta_T \sim N_p\left(m_t, C_t \right) \\
	%
	& \text{(3) For } t = T-1, \hdots, 0 \text{: Draw } \theta_t \sim N_p\left(h_t, H_t \right) \text{ where }\\
	%
	& \hspace{2 cm} h_t = m_t + C_t G_{t+1}' R_{t+1} ^{-1} \left(\theta_{t+1} - a_{t+1} \right) \\
	%
	& \hspace{2 cm} H_t = C_t - C_t G_{t+1}' R_{t+1} ^{-1} G_{t+1} C_t
\end{align*}

The Kalman Filter step is as follows. \\

Begin with $\theta_0 \sim N_p\left(m_0, C_0 \right)$. For $t \geq 1$, let $\left(\theta_{t-1} | y_{1:(t-1)} \right) \sim N_p\left(m_{t-1}, C_{t-1} \right)$.
\begin{align*}
	& \text{(i) The 1-step-ahead predictive distribution of } 
		\left(\theta_t | y_{1:(t-1)} \right) \sim N_p\left(a_t, R_t \right) \text{ with } \\
	%
	&\hspace{2 cm} a_t = E\left(\theta_t | y_{1:(t-1)} \right) = G_t m_{t-1} \\
	%
	&\hspace{2 cm} R_t = Var\left(\theta_t | y_{1:(t-1)} \right) = G_t C_{t-1}G_t' + W \\
	%
	%
	& \text{(ii) The 1-step-ahead predictive distribution of } \left( y_t | y_{1:(t-1)}\right) 
		\sim N_1\left(f_t, Q_t \right) \text{ with } \\
	%
	&\hspace{2 cm} f_t = E\left(y_t | y_{1:(t-1)} \right) = F_t a_t \\
	%
	&\hspace{2 cm} Q_t = Var\left(y_t | y_{1:(t-1)} \right) = F_t R_t F_t' + V \\
	%
	%
	& \text{(iii) The filtering distribution of } \left(\theta_t | y_{1:T} \right) \sim N_p\left(m_t, C_t \right) \text{ with } \\
	%
	&\hspace{2 cm} m_t = E\left(\theta_t | y_{1:T} \right) = a_t + R_t F_t' Q_t ^{-1} e_t \\
	%
	&\hspace{2 cm} C_t = Var\left(\theta_t | y_{1:T} \right) = R_t - R_t F_t' Q_t ^{-1} F_t R_T \\
	%
	&\hspace{2 cm} e_t = y_t - f_t \text{ (forecast error)}
\end{align*}

Notice that because we implement a Gibbs Sampler, we sample $V$ and $W$, and so the step to sample the states via FFBS treats $V$ and $W$ as known. Therefore our standard Gaussian Kalman Filter updates apply.\\

After estimating $V$, $W$, and the states $\theta_{1:T}$, we forecast future obserations at times $\left\{t+1, \hdots, t+K \right\}$. We set $K=100$, to estimate approximately the next four days of energy demand.  Begin by letting $a_t(0) = m_t$ and $R_t(0) = C_t$, where zero represents the current observed time.  Then for $k \geq 1$, the following recursions hold.
\begin{align*}
		&\text{(1) } \theta_{t+k} | y_{1:T} \sim N_p\left(a_t(k), R_t(k) \right) \text{, with }\\
		%
		&\hspace{2cm} a_t(k) = G_{t+k}a{t,k-1} \\
		%
		&\hspace{2cm} R_t(k) = G_{t+k} R{t,k-1} G_{t+k}' + W \\\\
		%
		&\text{(1) } y_{t+k} | y_{1:T} \sim N_1\left(f_t(k), Q_t(k) \right) \text{, with }\\
		%
		&\hspace{2cm} f_t(k) = F_{t+k} a_t(k) \\
		%
		&\hspace{2cm} Q_t(k) = F_{t+k} R_t(k) F_{t+k}' + V
\end{align*} 

%%%%%%%%%%%%%%%%%%%%%%%%%%%%%%%%%%%%%%%%%%%%%%%%%%%%%%%%%%%%%%%%%%%%
%%                     PART 3                                     %%
%%%%%%%%%%%%%%%%%%%%%%%%%%%%%%%%%%%%%%%%%%%%%%%%%%%%%%%%%%%%%%%%%%%%
\newpage
\section{3. Results}




%%%%%%%%%%%%%%%%%%%%%%%%%%%%%%%%%%%%%%%%%%%%%%%%%%%%%%%%%%%%%%%%%%%%
%%                     PART 4                                    %%
%%%%%%%%%%%%%%%%%%%%%%%%%%%%%%%%%%%%%%%%%%%%%%%%%%%%%%%%%%%%%%%%%%%%
\newpage
\section{4. Discussion}

%%%%%%%%%%%%%%%%%%%%%%%%%%%%%%%%%%%%%%%%%%%%%%%%%%%%%%%%%%%%%%%%%%%%
%%                     PART 5                                    %%
%%%%%%%%%%%%%%%%%%%%%%%%%%%%%%%%%%%%%%%%%%%%%%%%%%%%%%%%%%%%%%%%%%%%
\newpage
\section{5. References}

\begin{enumerate}[label={[\arabic*]}]
	\item West M, Harrison J. Baeysian Forecasting and Dynamic Models. 2nd ed. New York: Springer, 1997.
	\item Petris G, Petrone S, Campagnoli P. Dynamic Linear Models with R. New York: Springer Science + Business Media, 2009.
	\item Carter C, Robert K. 1994. On Gibbs sampling for state space models. Biometrika, 81, 541- 553.
\end{enumerate}



%###################################################################

\end{document}


